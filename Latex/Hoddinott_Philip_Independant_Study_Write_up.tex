%%%%%%%%%%%%%%%%%%%%%%%%%%%%%%%%%%%%%%%%%%%%%%%%%%%%%%%%%%%%
% Paul McKee
% Rensselaer Polytechnic Institute
% 1/31/18
% Master's Thesis
% with Dr. Kurt Anderson
% LaTeX Template: Project Titlepage Modified (v 0.1) by rcx
%%%%%%%%%%%%%%%%%%%%%%%%%%%%%%%%%%%%%%%%%%%%%%%%%%%%%%%%%%%%

\documentclass[12pt]{article}

%\usepackage[demo]{graphicx}
\usepackage{caption}
\usepackage{subcaption}

\usepackage{blindtext}
\usepackage[utf8]{inputenc}

\usepackage{graphicx, wrapfig, subcaption, setspace, booktabs}
\usepackage{sectsty}
\usepackage{url, lipsum}
\usepackage{makecell}
\usepackage{amsmath}
\usepackage{setspace}
\usepackage{amsmath}
\usepackage{color} %red, green, blue, yellow, cyan, magenta, black, white
\definecolor{mygreen}{RGB}{28,172,0} % color values Red, Green, Blue
\definecolor{mylilas}{RGB}{170,55,241}

\usepackage[table,xcdraw]{xcolor}


\usepackage[margin=.75in]{geometry} 
\usepackage{amsmath,amsthm,amssymb}
\usepackage{color}
\usepackage{fancyhdr}
\usepackage{lastpage}
\usepackage{graphicx}

\usepackage{cite}%AddedbyChris
\usepackage{graphicx}%AddedbyChris
\usepackage{array}%Added by Chris
\usepackage{caption}
\usepackage{amsmath}%Added by Chri
\pagestyle{fancy}
\fancyhf{}
%\fancyhead[L]{Independant Study}
%\fancyhead[C]{\rightmark}
\fancyhead[C]{}

\fancyhead[L]{\nouppercase{\leftmark}}
\fancyhead[R]{Philip Hoddinott}

\rfoot{Page \thepage \hspace{1pt} of \pageref{LastPage}}

\newcommand{\N}{\mathbb{N}}
\newcommand{\Z}{\mathbb{Z}}
\newcommand{\norm}[1]{\left\lVert#1\right\rVert}

\newenvironment{theorem}[2][Theorem]{\begin{trivlist}
		\item[\hskip \labelsep {\bfseries #1}\hskip \labelsep {\bfseries #2.}]}{\end{trivlist}}
\newenvironment{lemma}[2][Lemma]{\begin{trivlist}
		\item[\hskip \labelsep {\bfseries #1}\hskip \labelsep {\bfseries #2.}]}{\end{trivlist}}
\newenvironment{exercise}[2][Exercise]{\begin{trivlist}
		\item[\hskip \labelsep {\bfseries #1}\hskip \labelsep {\bfseries #2.}]}{\end{trivlist}}
\newenvironment{problem}[2][Problem]{\begin{trivlist}
		\item[\hskip \labelsep {\bfseries #1}\hskip \labelsep {\bfseries #2.}]}{\end{trivlist}}
\newenvironment{question}[2][Question]{\begin{trivlist}
		\item[\hskip \labelsep {\bfseries #1}\hskip \labelsep {\bfseries #2.}]}{\end{trivlist}}
\newenvironment{corollary}[2][Corollary]{\begin{trivlist}
		\item[\hskip \labelsep {\bfseries #1}\hskip \labelsep {\bfseries #2.}]}{\end{trivlist}}

\newenvironment{solution}{\begin{proof}[Solution]}{\end{proof}}
\usepackage{multicol}
\newcommand{\mysize}{0.5}
\usepackage{subcaption}
\usepackage{float}
\usepackage{listings}
\usepackage{color} 
\newcolumntype{L}{>{\centering\arraybackslash}m{3cm}}
\usepackage{setspace}
\usepackage[framed,numbered,autolinebreaks,useliterate]{mcode}

\newlength\longest

% %---------------------------------------------------------------
% % HEADER & FOOTER
% %---------------------------------------------------------------

%\fancyhf{}
%\pagestyle{fancy}
%\renewcommand{\headrulewidth}{0pt}
%\setlength\headheight{0pt}
%\fancyhead[L]{ Paul McKee }
%\fancyhead[R]{Rensselaer Polytechnic Institute}
%\cfoot{ \thepage\ } 


%--------------------------------------------------------------
% TITLE PAGE
%--------------------------------------------------------------
\iffalse
\begin{titlepage}
	\title{ 
		\LARGE \textbf{\uppercase{Put Title Here}} \\
		\vspace{0.25cm}
		\LARGE \textbf{Philip Hoddinott}
	}
	\author{\small{Submitted in Partial Fulfillment of the Requirements} \\ \small{for the Degree of} \\
		\uppercase{Master of Science} \\ \\
		Approved by:
		\\ Kurt Anderson, Chair \\ John Christian \\ Matthew Oehlschlaeger \\ \\ %% from paul's template
		\includegraphics[width=2.5cm]{rensselaer_seal.png} \\
		\small{\textit{Department of Mechanical, Aerospace, and Nuclear Engineering}} \\
		\small{Rensselaer Polytechnic Institute} \\ 
		\small{Troy, New York} \\
		\small{November 2018}
	}
\end{titlepage}
\fi

\begin{document}
		\title{A comparison of the Gibbs and Herrick-Gibbs Methods subject to noise for low eccentric orbits}
	\author{Philip Hoddinott}
	
	\maketitle
	\pagenumbering{roman}
	\thispagestyle{empty}
	\clearpage
	
	\thispagestyle{empty}
	\null\vfill
	
	\begin{figure}[!t]
		\begin{center}
			\settowidth\longest{\huge\itshape But, Captain, I cannot change;}
			\parbox{\longest}{%
				\raggedright{\huge\itshape%
					But, Captain, I cannot change the laws of physics\par\bigskip
				}   
				\raggedleft\Large{-Lt. Commander Montgomery ``Scotty" Scott}\par
				\raggedleft\Large{USS}\textit{ Enterprise}\par%
				
			}
		\end{center}
	\end{figure}
	
	
	\null\vfill
	
	\newpage
	%\setcounter{page}{1}
	\tableofcontents
	
	\newpage


	%\thispagestyle{fancy}
	
	%\addcontentsline{toc}{section}{\uppercase{Table of Contents}}
	%\listoftables
	%\addcontentsline{toc}{section}{\uppercase{List of Tables}}
	\listoffigures
	%\addcontentsline{toc}{section}{\uppercase{List of Figures}}
	% -----------------------------
	
	% ------------------------------------------------------------
	% Acknowledgement
	% ------------------------------------------------------------
	\newpage
	\pagenumbering{arabic}
		\doublespacing

	
	% ------------------------------------------------------------
	% Abstract 
	% ------------------------------------------------------------
	
	\newpage
	\section*{Abstract}
	The purpose of this report is to perform an analysis of the Gibbs and Herrick-Gibbs methods of orbital determination when subject to noise. The methods are explained and their derivations given. They are then computed for a sample orbit at various eccentricities. The Root Mean Square Error of both methods is obtained and the accuracy of the methods compared. \par 
	The author would like to express his gratitude to Professor John Christian for his suggestion of this project and his assistance with the methods of orbital determination through out the semester.\par 
	

	%\textcolor{red}{ Do More}
	% ------------------------------------------------------------
	% Introduction
	% ------------------------------------------------------------
	\newpage
	 % this should start the normal numberinbg

	\section{Introduction}
	\subsection{Background}
	Watching the skys at night, the ancient Greeks noticed that some stars seemed to move in fixed path across then night sky. They called these stars \textit{planetes}, meaning wanderers. The Greeks were not the first civilization to take notice of the planets, Babylonian astronomers were already recording the motion of the planets for five hundred years. But they were the first to develop methods of predicting the planet's motion. These methods were so advanced, that they would remain in use for over a thousand years until they were eclipsed the works of Copernicus\cite{lectureOnGreekAstro}. %, Kepler, Newton, and Euler\cite{lectureOnGreekAstro}. 
	
	%\textcolor{red}{Segway hre}
	Today the prediction of planet's orbits falls under what we would call orbital determination. Orbital determination is the estimation of an object's orbital elements from certain measurements of the object (position, velocity, angle), often taken from some ground station. Modern orbital determination employs various methods such as Lambert's method, Gauss' method, Gibbs' method, Herrick-Gibbs method. Some methods such as Gibb's and Herrick-Gibbs' require only three position measurements. Lambert's method requires two position and velocity measurements, and Gauss's method uses measurements of an object's right ascension and declination. A method of orbital determination is considered a preliminary orbit determination method if it just uses the equations of two body motion, and neglects other effects on the orbit.\par 
	
	
	
		%\textcolor{red}{Put a bit more on the other methods}
		\subsection{Problem Statement}
		Of the mentioned methods of preliminary orbit determination, Gibbs and Herrick-Gibbs are two commonly used ones. Gibbs method uses three position measurements that are spaced relatively far apart to solve for the orbital elements. Herrick-Gibbs uses three position measurements that are close together and a Taylor series to solve for the orbital elements. Herrick-Gibbs is recommended for position measurements that are less than 5 degrees apart, and Gibbs should be used for measurements more than 5 degrees apart\cite{vallado2007fundamentals}. However this suggestion does factor in error in the position measurements. Some work has been done on these three position measurement methods \cite{Kaushick}, in this paper we seek to expand on it and add in the effects of orbital eccentricity.
		\subsection{Transition Line}
		Kaushick \cite{Kaushick} in his comparison of the Gibbs and Herrick-Gibbs methods discusses the idea of a transition point, the place where on a plot of error vs the separation of position measurements that the Gibbs method becomes more accurate than the Herrick-Gibbs method.\par 
		In this paper the idea of a transition line will be used. The transition line is the three dimensional representation of the transition point. The transition line marks where on a range of noise and measurement separations the Gibbs method becomes better than the Herrick-Gibbs method.
		
		
		

	%\subsection{Orbit Determination}
	%
	%\newpage
	\section{Gibbs method}
	Gibbs method was developed by Josiah Willard Gibbs\cite{gibbsBio}, who was famous for his application of vectorized math to thermodynamics. Gibbs method works off having three measurements of the distance to an object at three different times. Thus, assume we have the following position vectors:
	\begin{equation}
	\boldsymbol{r_1}, 	\boldsymbol{r_2},	\boldsymbol{r_3}
	\end{equation}
	which were observed at times $t_1, t_2, \text{ and  } t_3$. Each of these position vectors has a corresponding velocity vector $\boldsymbol{v_1}, 	\boldsymbol{v_2},	\text{ and  } \boldsymbol{v_3}$. To obtain the orbital elements from one of the position vectors, the corresponding velocity vector is needed. Thus the focus of Gibbs method is searching for a velocity vector. This velocity vector may be found in just four equations:
	\begin{equation}
	\mathbf { N } = r _ { 1 } \left( \mathbf { r } _ { 2 } \times \mathbf { r } _ { 3 } \right) + r _ { 2 } \left( \mathbf { r } _ { 3 } \times \mathbf { r } _ { 1 } \right) + r _ { 3 } \left( \mathbf { r } _ { 1 } \times \mathbf { r } _ { 2 } \right)
	\end{equation}
	\begin{equation}
	\mathbf { D } = \mathbf { r } _ { 1 } \times \mathbf { r } _ { 2 } + \mathbf { r } _ { 2 } \times \mathbf { r } _ { 3 } + \mathbf { r } _ { 3 } \times \mathbf { r } _ { 1 }
	\end{equation}
	\begin{equation}
	\mathbf { S } = \mathbf { r } _ { 1 } \left( r _ { 2 } - r _ { 3 } \right) + \mathbf { r } _ { 2 } \left( r _ { 3 } - r _ { 1 } \right) + \mathbf { r } _ { 3 } \left( r _ { 1 } - r _ { 2 } \right)
	\end{equation}
	\begin{equation}
	\mathbf { v } = \sqrt { \frac { \mu } { N D } } \left( \frac { \mathbf { D } \times \mathbf { r } } { r } + \mathbf { S } \right)
	\end{equation}
	
	This algorithm works well for a code to obtain orbital elements, and a MATLAB code for it is provided in the appendix.
		
	\subsection{Derivation of Gibbs Method}
	The algorithm above will now be expanded on, to explain how the steps were obtained. We will use the notation from Howard Curtis's Orbital Mechanics for Engineering Students \cite{curtis2013_gibbs}. 
	
	Gibbs method beings with the conservation of momentum, which means that all the position vectors of an orbiting object must all be co planer. Or the unit vector normal to the plane of any two position vectors must be perpendicular to the unit vector of the third measurement. This may be expressed as:
	
	\begin{eqnarray}
	\hat { { u } } _ { r 1 } = \mathbf { r } _ { 1 } / r _ { 1 }& \hat { \mathrm { C } } _ { 23 } = \left( \mathbf { r } _ { 2 } \times \mathbf { r } _ { 3 } \right) / \left\| \mathbf { r } _ { 2 } \times \mathbf { r } _ { 3 } \right\|\\
	\hat {{ u } } _ { r 2 } = \mathbf { r } _ { 2 } / r _ { 2 }&\hat { \mathrm { C } } _ { 31 } = \left( \mathbf { r } _ { 3 } \times \mathbf { r } _ { 1} \right) / \left\| \mathbf { r } _ { 3 } \times \mathbf { r } _ { 1 } \right\|\\
	\hat {{ u } } _ { r 3 } = \mathbf { r } _ { 3 } / r _ { 3} & \hat { \mathrm { C } } _ { 12 } = \left( \mathbf { r } _ { 1 } \times \mathbf { r } _ { 2 } \right) / \left\| \mathbf { r } _ { 1} \times \mathbf { r } _ { 2 } \right\|
	\end{eqnarray}
	%\textcolor{red}{toDO, make these normal fractions, i like those better}
	
	To check that all the vectors are co planer as they should be, the dot product of 
	
	\begin{equation}
	\hat {  { u } } _ { r 1 } \cdot \hat {  { C } } _ { 23 } = 0
	\end{equation}
	
	Additionally because the vectors are co planer, there must be scalar factors $(c)$ that can be applied to two of the vectors to make their sum equal to the third vector:
	
	\begin{equation}
	\mathbf{r}_2 = c_1 \mathbf{ r }_1 + c_3 \mathbf{r}_3
	\label{eqn:cs}
	\end{equation}
	We should note that convention has Gibbs method solve for the velocity vector associated with the second position vector. Gibbs method can be used to solve for the other vectors, the order of the position vectors must simply be rotated.
	
		
	If these vectors are all co planer, we may define unit vector $\hat{\mathbf{w}}$ as a unit vector normal to the orbital plane, unit vector $\hat{\mathbf{p}}$ as being in the direction of the eccentricity vector, and $\hat{\mathbf{q}}$ as the unit vector that is normal to $\hat{\mathbf{w}}$ and $\hat{\mathbf{p}}$, such that:
	\begin{equation}
	\hat{\mathbf{q}}=\hat{\mathbf{w}}\times\hat{\mathbf{p}}
	\label{eqn:cross_wpq}
	\end{equation}
	The eccentricity and angular moment vectors may be written as 
	\begin{eqnarray}
	\mathbf{h}=h\hat{\mathbf{w}}\\
	\mathbf{e}=e\hat{\mathbf{p}}
	\end{eqnarray}
	
	\begin{equation}
	\mathbf { v } \times \mathbf { h } = \mu \left( \frac { \mathbf { r } } { r } + \mathbf { e } \right)
	\label{eqn:Og_h}
	\end{equation}
	
	The relationship between position, velocity, momentum, and eccentricity may be expressed as equation \ref{eqn:Og_h}. We rewrite equation \ref{eqn:Og_h} to solve for velocity as equation \ref{eqn:new_h}
	
	\begin{equation}
	\mathbf { v } = \frac { \mu } { h ^ { 2 } } \left( \frac { \mathbf { h } \times \mathbf { r } } { r } + \mathbf { h } \times \mathbf { e } \right)
	\label{eqn:new_h}
	\end{equation}

	
		and using equation \ref{eqn:cross_wpq}, equation \ref{eqn:new_h} may be rewritten as
	 \begin{equation}
	 \mathbf { v } = \frac { \mu } { h } \left( \frac { \hat { \mathbf { w } } \times \mathbf { r } } { r } + e \hat { \mathbf { q } } \right)
	 \label{eqn:almostDone}
	 \end{equation}
	 
	 The relationship between eccentricity, angular momentum, and position vector may be written as 
	 \begin{equation}
	 \mathbf { r } \cdot \mathbf { e } = \frac { h ^ { 2 } } { \mu } - r 
	 \label{eqn:eampo}
	 \end{equation}
	 Equation \ref{eqn:eampo} has a relation of position, eccentricity, and momentum. To eliminat eccentricity, we first dot equation \ref{eqn:cs} with the eccentricity vector to get equation \ref{eqn:edot} and then insert equation \ref{eqn:eampo} into equation \ref{eqn:edot}.
	 
	 \begin{equation}
	 \left(\mathbf{r}_2 \right)\cdot\mathbf{ e }= \left(c_1 \mathbf{ r }_1 + c_3 \mathbf{r}_3\right)\cdot\mathbf{ e }=c_1 \mathbf{ r }_1\cdot\mathbf{ e } + c_3 \mathbf{r}_3\cdot\mathbf{ e }
	 \label{eqn:edot}
	 \end{equation}
	 
	 
	 \begin{equation}
	 \left(\frac { h ^ { 2 } } { \mu } - r_2\right) =c_1 \left(\frac { h ^ { 2 } } { \mu } - r_1\right)  + c_3\left(\frac { h ^ { 2 } } { \mu } - r _3\right)
	 \label{eqn:edot2}
	 \end{equation}
	 
	 Now we have a relationship between the coefficients from earlier, position, and momentum. If the coefficients $c_1$ and $c_2$ may be solved for, the angular momentum may be found. 
	 
	 Removing the coefficients gives the following equation
	 \begin{equation}
	 \frac { h ^ { 2 } } { \mu } \underbrace{\left( \mathbf { r } _ { 1 } \times \mathbf { r } _ { 2 } + \mathbf { r } _ { 2 } \times \mathbf { r } _ { 3 } + \mathbf { r } _ { 3 } \times \mathbf { r } _ { 1 } \right)}_{D} = \underbrace{r _ { 1 } \left( \mathbf { r } _ { 2 } \times \mathbf { r } _ { 3 } \right) + r _ { 2 } \left( \mathbf { r } _ { 3 } \times \mathbf { r } _ { 1 } \right) + r _ { 3 } \left( \mathbf { r } _ { 1 } \times \mathbf { r } _ { 2 } \right)}_{N}
	 \label{eqn:longone}
	 \end{equation}
	 For simplicity we create the vectors N and D, which are: 
	 \begin{eqnarray}
	 	 \mathbf { N } = r _ { 1 } \left( \mathbf { r } _ { 2 } \times \mathbf { r } _ { 3 } \right) + r _ { 2 } \left( \mathbf { r } _ { 3 } \times \mathbf { r } _ { 1 } \right) + r _ { 3 } \left( \mathbf { r } _ { 1 } \times \mathbf { r } _ { 2 } \right)\\
	 	 \mathbf { D } = \mathbf { r } _ { 1 } \times \mathbf { r } _ { 2 } + \mathbf { r } _ { 2 } \times \mathbf { r } _ { 3 } + \mathbf { r } _ { 3 } \times \mathbf { r } _ { 1 }\\
	 	 N=\norm{\mathbf{N}}\\
	 	 D=\norm{\mathbf{D}}
	 \end{eqnarray}
 	Now using this notation equation \ref{eqn:longone} may be rewritten as 
	\begin{equation}
	\mathbf { N } = \frac { h ^ { 2 } } { \mu } \mathbf { D }
	\end{equation}
	Replacing the vectors with their norms and rearranging:
	\begin{eqnarray}
	N = \frac { h ^ { 2 } } { \mu } D\\
	h = \sqrt { \mu \frac { N } { D } }
	\label{eqn:gotH}
	\end{eqnarray}
	From equation \ref{eqn:gotH} the angular momentum may be found from just the position vectors.
	
	Because $\hat{ \mathbf { w } }$ was defined as being a unit vector normal to the orbital plane, and $\mathbf{ N }$ and $\mathbf{ D }$ are made up of vectors in the orbital plane then
	\begin{equation}
	\hat{ \mathbf { w } }=\frac{\mathbf{ N }}{N}=\frac{\mathbf{ D }}{D}
	\end{equation}
	
	Equation \ref{eqn:cross_wpq} may then be written as
	\begin{equation}
	\hat { \mathbf { q } } = \frac { ( \mathbf { D } \times \mathbf { e } ) } { D e }
	\label{eqn:p1}
	\end{equation}
	Fully writing out the numerator:
	
	\begin{equation}
	\hat { \mathbf { q } } = \frac { \left[ \left( \mathbf { r } _ { 1 } \times \mathbf { r } _ { 2 } \right) \times \mathbf { e } + \left( \mathbf { r } _ { 2 } \times \mathbf { r } _ { 3 } \right) \times \mathbf { e } + \left( \mathbf { r } _ { 3 } \times \mathbf { r } _ { 1 } \right) \times \mathbf { e } \right] } { D e } 
	\end{equation}
	
	Using the bac-cab vector identity, the following equations are obtained
	\begin{equation}
	( \mathbf { A } \times \mathbf { B } ) \times \mathbf { C }  = \mathbf { B } ( \mathbf { A } \cdot \mathbf { C } ) - \mathbf { A } ( \mathbf { B } \cdot \mathbf { C } )
	\end{equation}
	\begin{eqnarray}
	{ \left( \mathbf { r } _ { 2 } \times \mathbf { r } _ { 3 } \right) \times \mathbf { e } = \mathbf { r } _ { 3 } \left( \mathbf { r } _ { 2 } \cdot \mathbf { e } \right) - \mathbf { r } _ { 2 } \left( \mathbf { r } _ { 3 } \cdot \mathbf { e } \right) } \\ { \left( \mathbf { r } _ { 3 } \times \mathbf { r } _ { 1 } \right) \times \mathbf { e } = \mathbf { r } _ { 1 } \left( \mathbf { r } _ { 3 } \cdot \mathbf { e } \right) - \mathbf { r } _ { 3 } \left( \mathbf { r } _ { 1 } \cdot \mathbf { e } \right) } \\ { \left( \mathbf { r } _ { 1 } \times \mathbf { r } _ { 2 } \right) \times \mathbf { e } = \mathbf { r } _ { 2 } \left( \mathbf { r } _ { 1 } \cdot \mathbf { e } \right) - \mathbf { r } _ { 1 } \left( \mathbf { r } _ { 2 } \cdot \mathbf { e } \right) } 
	\label{eqn:usedVectId}
	\end{eqnarray}
	Inserting equation \ref{eqn:eampo} into equation \ref{eqn:usedVectId}  
	
	\begin{eqnarray}
	\left( \mathbf { r } _ { 2 } \times \mathbf { r } _ { 3 } \right) \times \mathbf { e } = \mathbf { r } _ { 3 } \left( \frac { h ^ { 2 } } { \mu } - r _ { 2 } \right) - \mathbf { r } _ { 2 } \left( \frac { h ^ { 2 } } { \mu } - r _ { 3 } \right) = \frac { h ^ { 2 } } { \mu } \left( \mathbf { r } _ { 3 } - \mathbf { r } _ { 2 } \right) + r _ { 3 } \mathbf { r } _ { 2 } - r _ { 2 } \mathbf { r } _ { 3 }\\
	\left( \mathbf { r } _ { 3 } \times \mathbf { r } _ { 1 } \right) \times \mathbf { e } = \mathbf { r } _ { 1 } \left( \frac { h ^ { 2 } } { \mu } - r _ { 3 } \right) - \mathbf { r } _ { 3 } \left( \frac { h ^ { 2 } } { \mu } - r _ { 1 } \right) = \frac { h ^ { 2 } } { \mu } \left( \mathbf { r } _ { 1 } - \mathbf { r } _ { 3 } \right) + r _ { 1 } \mathbf { r } _ { 3 } - r _ { 3 } \mathbf { r } _ { 1 }\\
	\left( \mathbf { r } _ { 1 } \times \mathbf { r } _ { 2 } \right) \times \mathbf { e } = \mathbf { r } _ { 2 } \left( \frac { h ^ { 2 } } { \mu } - r _ { 1 } \right) - \mathbf { r } _ { 1 } \left( \frac { h ^ { 2 } } { \mu } - r _ { 2 } \right) = \frac { h ^ { 2 } } { \mu } \left( \mathbf { r } _ { 2 } - \mathbf { r } _ { 1 } \right) + r _ { 2 } \mathbf { r } _ { 1 } - r _ { 1 } \mathbf { r } _ { 2 }
	\end{eqnarray}
	
%\textcolor{red}{	NOte take out the midle part}

The we write the sum of these equations as 

	\begin{equation}
	\mathbf { S } = \mathbf { r } _ { 1 } \left( r _ { 2 } - r _ { 3 } \right) + \mathbf { r } _ { 2 } \left( r _ { 3 } - r _ { 1 } \right) + \mathbf { r } _ { 3 } \left( r _ { 1 } - r _ { 2 } \right)
	\end{equation}
	And equation \ref{eqn:p1} can now be written as 
	\begin{equation}
	\hat { \mathbf { q } } = \frac { 1 } { D e } \mathbf { S }
	\label{eqn:ad2}
	\end{equation}
	Inserting equation \ref{eqn:ad2}into equation \ref{eqn:almostDone} yields equation \ref{eqn:FinalP}. 
		
	\begin{equation}
	\mathbf { v } = \frac { \mu } { h } \left( \frac { \hat { \mathbf { w } } \times \mathbf { r } } { r } + e \hat { \mathbf { q } } \right)=\frac { \mu } { \sqrt { \mu \frac { N } { D } } } \left[ \frac { \frac { D } { D } \times \mathbf { r } } { r } + e \left( \frac { 1 } { D e } \mathbf { S } \right) \right]
	\label{eqn:FinalP}
	\end{equation}
	
	Equation \ref{eqn:FinalP} may be simplified to \ref{eqn:f21}
	
	\begin{equation}
	\mathbf { v } = \sqrt { \frac { \mu } { N D } } \left( \frac { \mathbf { D } \times \mathbf { r } } { r } + \mathbf { S } \right)
	\label{eqn:f21}
	\end{equation}
	 Now from this equation the velocity vector may be computed. From the velocity and position vector, the orbital elements may be computed. 
	\iffalse
	\subsubsection{Gibbs method Algorithm}
	These steps follow the derication og gibbs method from Orbital Mechanics for Engineering Students \cite{curtis2013_gibbs}. A more detailed derivation may be found in the appendix
	
	\begin{equation}
	\mathbf { N } = r _ { 1 } \left( \mathbf { r } _ { 2 } \times \mathbf { r } _ { 3 } \right) + r _ { 2 } \left( \mathbf { r } _ { 3 } \times \mathbf { r } _ { 1 } \right) + r _ { 3 } \left( \mathbf { r } _ { 1 } \times \mathbf { r } _ { 2 } \right)
	\end{equation}
	
	and
	\begin{equation}
	\mathbf { D } = \mathbf { r } _ { 1 } \times \mathbf { r } _ { 2 } + \mathbf { r } _ { 2 } \times \mathbf { r } _ { 3 } + \mathbf { r } _ { 3 } \times \mathbf { r } _ { 1 }
	\end{equation}
	
	\begin{equation}
	\mathbf { S } = \mathbf { r } _ { 1 } \left( r _ { 2 } - r _ { 3 } \right) + \mathbf { r } _ { 2 } \left( r _ { 3 } - r _ { 1 } \right) + \mathbf { r } _ { 3 } \left( r _ { 1 } - r _ { 2 } \right)
	\end{equation}
	From these
	\begin{equation}
	\mathbf { v } = \sqrt { \frac { \mu } { N D } } \left( \frac { \mathbf { D } \times \mathbf { r } } { r } + \mathbf { S } \right)
	\end{equation}
	
	From this velocity, the oribtal elements may be found
	
	
	Write up
	Page or two on orbit degetrimantion
	Angle only mtheod: gauss, ;laplace
	Velotto textbook on Orbit deterimantion
	3 pos vect, 3 beartin time
	Thrre velc , IOD two
	(no eqn)
	Section on gibbs
	Frame problem
	Derive gibbs
	Show how you go through problem set up to the end
	Talk about now, assses how well gibbs does for ODE, diff ecc, diff noise level
	Talk about results
	Derivation of gibbs should be a page to two pages (plus figures)
	Harrit gibbs method (try to take a look at this) (it’s a taylor series expansion of orbit around)
	Nice plots
	Get a pdf of report by last day of class
	Mean  anonly, not true for circ
	
	
	\subsection{Derivation of Gibbs Method}
	\begin{eqnarray}
	\hat { { u } } _ { r 1 } = \mathbf { r } _ { 1 } / r _ { 1 }& \hat { \mathrm { C } } _ { 23 } = \left( \mathbf { r } _ { 2 } \times \mathbf { r } _ { 3 } \right) / \left\| \mathbf { r } _ { 2 } \times \mathbf { r } _ { 3 } \right\|\\
	\hat {{ u } } _ { r 2 } = \mathbf { r } _ { 2 } / r _ { 2 }&\hat { \mathrm { C } } _ { 31 } = \left( \mathbf { r } _ { 3 } \times \mathbf { r } _ { 1} \right) / \left\| \mathbf { r } _ { 3 } \times \mathbf { r } _ { 1 } \right\|\\
		\hat {{ u } } _ { r 3 } = \mathbf { r } _ { 3 } / r _ { 3} & \hat { \mathrm { C } } _ { 12 } = \left( \mathbf { r } _ { 1 } \times \mathbf { r } _ { 2 } \right) / \left\| \mathbf { r } _ { 1} \times \mathbf { r } _ { 2 } \right\|
	\end{eqnarray}
	NOTE make these normal fractions
	\fi
	\section{Herrick-Gibbs Method}
	The obvious problem with Gibbs method is what if the measured vectors are close to each other, for example in a single pass over a ground station. Herrick Gibbs is one such solution to this problem, it is a method of orbital determination that requires three position measurements, and their respective times. It then uses a Taylor series expansion. Because it is a Taylor series, it is not as robust as Gibbs method. According to Vallado, it is best for under 5 degrees of separation between position vectors \cite{vallado2007fundamentals}.
	
	\subsection{Herrick-Gibbs Method Algorithm}
	The derivation of Herrick-Gibbs method is a lengthy one, and readers searching for a more in depth derivation of Herrick-Gibbs should read Vallado's Fundamentals of Astrodynamics and Applications \cite{vallado2007fundamentals}. This paper will present the relevant algorithm for Herrick Gibbs method. The algorithm of the Herrick-Gibbs method in this section is based off of Kaushick's derivation\cite{Kaushick}, but the notation has been changed to be consistent with the notation used thus far in this paper. Herrick Gibbs begins by using the Taylor series to expand the position and the velocity vectors out by 
	\begin{equation}
	\begin{array} { l } { \mathbf { r } = \mathbf { a } _ { 0 } + t \mathbf { a } _ { 1 } + t ^ { 2 } \mathbf { a } _ { 2 } + t ^ { 3 } \mathbf { a } _ { 3 } + t ^ { 4 } \mathbf { a } _ { 4 } + t ^ { 5 } \mathbf { a } _ { 5 } } \\ { \mathbf { v } = \mathbf { a } _ { 1 } + 2 t \mathbf { a } _ { 2 } + 3 t ^ { 2 } \mathbf { a } _ { 3 } + 4 t ^ { 3 } \mathbf { a } _ { 4 } + 5 t ^ { 4 } \mathbf { a } _ { 5 } } \end{array}
	\end{equation}
	Where the $a_n$ terms are unknowns from the power series. The equation for acceleration in a two body system then becomes
	\begin{equation}
	- \frac { \mu } { r ^ { 3 } } \mathbf { r } = 2 \mathbf { a } _ { 2 } + 6 t \mathbf { a } _ { 3 } + 12 t ^ { 2 } \mathbf { a } _ { 4 } + 20 t ^ { 3 } \mathbf { a } _ { 5 }
	\end{equation}
	The time between the observations is defined as 
	\begin{equation}
	\begin{array} { l } { \Delta t _ { 32 } = t _ { 3 } - t _ { 2 } } \\ { \Delta t _ { 31 } = t _ { 3 } - t _ { 1 } } \\ { \Delta t _ { 21 } = t _ { 2 } - t _ { 1 } } \end{array}
	\end{equation}
	The position vectors may be now written as 
	\begin{equation}
	\begin{array} { l }
	{ \mathbf { r } _ { 2 } = \mathbf { a } _ { 0 } } \\ 
	 { \mathbf { r } _ { 1 } = \mathbf { r } _ { 2 } - \Delta t _ { 21 } \mathbf { a } _ { 1 } + \Delta t _ { 21 } ^ { 2 } \mathbf { a } _ { 2 } - \Delta t _ { 21 } ^ { 3 } \mathbf { a } _ { 3 } + \Delta t _ { 12 } ^ { 4 } \mathbf { a } _ { 4 } - \Delta t _ { 21 } ^ { 5 } \mathbf { a } _ { 5 } } \\ 
	
	{ \mathbf { r } _ { 3 } = \mathbf { r } _ { 2 } - \Delta t _ { 32 } \mathbf { a } _ { 1 } + \Delta t _ { 32 } ^ { 2 } \mathbf { a } _ { 2 } - \Delta t _ { 32 } ^ { 3 } \mathbf { a } _ { 3 } + \Delta t _ { 32 } ^ { 4 } \mathbf { a } _ { 1 } - \Delta t _ { 32 } ^ { 5 } \mathbf { a } _ { 5 } } \\ 	
	\end{array}
	\end{equation}
	The velocity and acceleration vectors may be now written as
	 \begin{equation}
	 \begin{aligned} \mathbf { v } _ { 2 } & = \mathbf { a } _ { 1 } \\ - \frac { \mu } { r _ { 1 } ^ { 3 } } \mathbf { r } _ { 1 } & = 2 \mathbf { a } _ { 2 } - 6 \Delta t _ { 21 } \mathbf { a } _ { 3 } + 12 \Delta t _ { 21 } ^ { 2 } \mathbf { a } _ { 4 } - 20 \Delta t _ { 21 } ^ { 3 } \mathbf { a } _ { 5 } \\ - \frac { \mu } { r _ { 2 } ^ { 3 } } \mathbf { r } _ { 2 } & = 2 \mathbf { a } _ { 2 } \\ - \frac { \mu } { r _ { 3 } ^ { 3 } } \mathbf { r } _ { 3 } & = 2 \mathbf { a } _ { 2 } + 6 \Delta t _ { 32 } \mathbf { a } _ { 3 } + 12 \Delta t _ { 32 } ^ { 2 } \mathbf { a } _ { 4 } + 20 \Delta t _ { 32 } ^ { 3 } \mathbf { a } _ { 5 } \end{aligned}
	 \end{equation}
	
	Solving for the $a_n$ terms gives the following equation
	
	\begin{equation}
	\begin{aligned} \mathbf { v } _ { 2 } = - \Delta t _ { 32 } \left[ \frac { 1 } { \Delta t _ { 21 } \Delta t _ { 31 } } + \frac { \mu } { 12 r _ { 1 } ^ { 3 } } \right] \mathbf { r } _ { 1 } + \left( \Delta t _ { 32 } - \Delta t _ { 21 } \right) & \left[ \frac { 1 } { \Delta t _ { 21 } \Delta t _ { 32 } } + \frac { \mu } { 12 r _ { 2 } ^ { 3 } } \right] \mathbf { r } _ { 2 } \\ + \Delta t _ { 21 } & \left[\frac { 1 } { \Delta t _ { 32 } \Delta t _ { 31 } } + \frac { \mu } { 12 r _ { 3 } ^ { 3 } } \right] \mathbf { r } _ { 3 } \end{aligned}
	\label{eqn:HGREF}
	\end{equation}
	
		%\begin{equation}
	%\mathbf { v } _ { 2 } = - \Delta t _ { 32 } \left[ \frac { 1 } { \Delta t _ { 21 } \Delta t _ { 31 } } + \frac { \mu } { 12 r _ { 1 } ^ { 3 } } \right] \mathbf { r } _ { 1 } + \left( \Delta t _ { 32 } - \Delta t _ { 21 } \right)  \left[ \frac { 1 } { \Delta t _ { 21 } \Delta t _ { 32 } } + \frac { \mu } { 12 r _ { 2 } ^ { 3 } } \right] \mathbf { r } _ { 2 } \\ + \Delta t _ { 21 }  \left[\frac { 1 } { \Delta t _ { 32 } \Delta t _ { 31 } } + \frac { \mu } { 12 r _ { 3 } ^ { 3 } } \right] \mathbf { r } _ { 3 } 
	%\end{equation}
	
	\iffalse
	
	%\textcolor{red}{To do, expand on this derivation, reverance Vallado}
	
	The derivation of the Herrick-Gibbs method in this section is based off of Vallado's derivation \cite{vallado2007fundamentals}, but the notation has been changed to be consistent with the notation used thus far in this paper. 
	
		Herrick Gibbs begins by using the Taylor series to expand the position vector out by 
		\begin{equation}
		\mathbf{r(t)}=\mathbf{r_2}+\mathbf{\dot{r}_2} (t-t_2) + \mathbf{\ddot{r}_2} \frac{(t-t_2)^2}{2!}+ \mathbf{\dddot{r}_2} \frac{(t-t_2)^3}{3!}+ \mathbf{{r_2}}^{(iv)} \frac{(t-t_2)^4}{4!}+\dots
		\end{equation}
		
		The time between the observations is defined as 
		\begin{equation}
		\begin{array} { l } { \Delta t _ { 23 } = t _ { 2 } - t _ { 3 } } \\ { \Delta t _ { 13 } = t _ { 1 } - t _ { 3 } } \\ { \Delta t _ { 12 } = t _ { 1 } - t _ { 12 } } \end{array}
		\end{equation}
		The position vectors may be now written as 
		
		\begin{equation}\begin{aligned}
		\mathbf{r_1}=\mathbf{r_2}+\mathbf{\dot{r}_2} (\Delta t _ { 12 } ) + \mathbf{\ddot{r}_2} \frac{(\Delta t _ { 12 })^2}{2!}+ \mathbf{\dddot{r}_2} \frac{(\Delta t _ { 12 })^3}{3!}+ \mathbf{{r_2}}^{(iv)} \frac{(\Delta t _ { 12 })^4}{4!}+\dots\\
		\mathbf{r_3}=\mathbf{r_2}+\mathbf{\dot{r}_2} (\Delta t _ { 32 } ) + \mathbf{\ddot{r}_2} \frac{(\Delta t _ { 32 })^2}{2!}+ \mathbf{\dddot{r}_2} \frac{(\Delta t _ { 32 })^3}{3!}+ \mathbf{{r_2}}^{(iv)} \frac{(\Delta t _ { 32 })^4}{4!}+\dots\end{aligned}
		\end{equation}
		Using the relations in \ref{eqn:timer} and multiplying the equations together yields
		\begin{equation}
		\begin{array} { l } { \Delta t _ { 12 } ^ { 2 } \Delta t _ { 32 } - \Delta t _ { 32 } ^ { 2 } \Delta t _ { 12 } = \Delta t _ { 12 } \Delta t _ { 32 } \Delta t _ { 13 } } \\ { \Delta t _ { 12 } ^ { 2 } \Delta t _ { 32 } ^ { 3 } - \Delta t _ { 32 } ^ { 2 } \Delta t _ { 12 } ^ { 3 } = \Delta t _ { 12 } ^ { 2 } \Delta t _ { 32 } ^ { 2 } \Delta t _ { 31 } } \\ { \Delta t _ { 12 } ^ { 2 } \Delta t _ { 32 } ^ { 4 } - \Delta t _ { 32 } ^ { 2 } \Delta t _ { 12 } ^ { 4 } = \Delta t _ { 12 } ^ { 2 } \Delta _ { 32 } ^ { 2 } \Delta t _ { 31 } \left\( \Delta t _ { 32 } + \Delta t _ { 12 } \right\) } \end{array}\label{eqn:timer}
		\end{equation}
		

		\begin{equation}
		\begin{aligned}
		\mathbf{\dot{r}_2} \left(\Delta t_{12}\Delta t_{32}\Delta t_{31}\right)=\mathbf{r_1}\Delta t_{32}^2 + \mathbf{r_2}\left(-\Delta t_{32}^2 +\Delta t_{12}^2 \right)-\mathbf{r_3}\Delta t_{12}^2\\
		+ \frac { \mathbf{\dot { r_2 }} } { 6 } \left( \Delta t _ { 12 } ^ { 2 } \Delta t _ { 32 } ^ { 2 } \Delta t _ { 31 } \right) + \frac { \mathbf{\dot { r } _ { 2 }} } { 24 } \left( \Delta t _ { 12 } ^ { 2 } \Delta t _ { 32 } ^ { 2 } \Delta t _ { 31 } \left\{ \Delta t _ { 32 } + \Delta t _ { 12 } \right\} \right)
		\end{aligned}
		\end{equation}

	
	
	Herrick Gibbs begins by using the Taylor series to expand the position and the velocity vectors out by 
	\begin{equation}
	\begin{array} { l } { \mathbf { r } = \mathbf { a } _ { 0 } + t \mathbf { a } _ { 1 } + t ^ { 2 } \mathbf { a } _ { 2 } + t ^ { 3 } \mathbf { a } _ { 3 } + t ^ { 4 } \mathbf { a } _ { 4 } + t ^ { 5 } \mathbf { a } _ { 5 } } \\ { \mathbf { v } = \mathbf { a } _ { 1 } + 2 t \mathbf { a } _ { 2 } + 3 t ^ { 2 } \mathbf { a } _ { 3 } + 4 t ^ { 3 } \mathbf { a } _ { 4 } + 5 t ^ { 4 } \mathbf { a } _ { 5 } } \end{array}
	\end{equation}
	
	
	
%\iffalse
	
	difference in time for the 
	\fi
	\section{Analysis of Methods}
	\subsection{Analysis Process}
	%\textcolor{red}{To Do: Put in Orbital Parameters for the orbit I am using}
	The orbits that were analyzed came from the following orbital elements:
	\begin{itemize}\singlespacing
		\item Inclination (i) of 30 degrees.
		\item Longitude of the ascending node $(\Omega)$ of 40 degrees.
		\item Argument of periapsis $(\omega)$ of 70 degrees.
		\item A periapsis of 7178.1 km. 
		\item An eccentricity that was varied.
		\item A true anomaly that was varied.
	\end{itemize}\doublespacing
	From these elements the semi major axis and the angular momentum was calculated to get the full set of orbital elements. 
	
	The first parameter investigated was the separation of position measurements.  Vallado\cite{vallado2007fundamentals} states that Herrick-Gibbs is only accurate for measurements taken within 5 degrees of the mean anomaly of eachother, however that seems dependent on other factors. The other parameter was the noise, how accurate the position measurements were. Cases for a range of eccentricity were run. The processes for each test goes as follows:
	\begin{enumerate}
		\item The eccentricity is set and the orbital parameters calculated.
		\item The distance between measurements measured in degrees is selected from a range. The exact position vectors are calculated.
		\begin{enumerate}
			\item The noise level is selected from a range. 
			\begin{enumerate}
				\item A large number of position vectors are generated from the exact position vector, and the noise level.
				\item The code uses a method of orbital determination to compute the semi major axis corresponding to each ``noisy" position vector. 
				\item The Root Mean Square error for the computed semi major axis vs the real semi major axis is calculated. 

			\end{enumerate}
				\item The next noise level is selected and the code repeats.
		\end{enumerate}
		\item The next distance between position measurements is selected, and the code repeats.
	\end{enumerate}

	The accuracy of two methods is compared by the Root Mean Square Error of their semi major axis. The Root Mean Square Error is computed by equation \ref{eqn:rmse}
	\begin{equation}
	RMSE(\hat{y})=\sqrt{E\left(\left(\hat{y}-y\right)^2\right)}
	\label{eqn:rmse}
	\end{equation}
	Where E is the expected value or the mean, $\hat{y}-y$ is the difference between the expected value and the measured value. 
	Figure \ref{fig:RMS_MATLAB} shows how this is implemented in MATLAB. 
		\begin{figure}[H]
		%\singlespacing
		\begin{lstlisting}
		RMSE = sqrt(mean((yhat - y).^2));
		\end{lstlisting}
		\caption{Root Mean Square Error Implemented in MATLAB}
		\label{fig:RMS_MATLAB}
		%\doublespacing
	\end{figure}
	
	The code that was used to to compare the methods and generate the plots may be seen in the appendix. As hundreds of orbits were compared not all plots can be reasonably fit into this report. Thus only the most relevant figures have been included. The additional plots and their data may be downloaded at the author's website: \url{rpi.edu\~hoddip}.
	\subsection{Analysis of a circular orbit}
	The first orbit that was analyzed was a circular orbit using the orbital elements previously listed. The RMS error for Gibbs methods may be seen in figure \ref{fig:circulargibbs}.
	\begin{figure}[H]
		\centering
		\includegraphics[width=0.7\linewidth]{circularGibbs}
		\caption{RMS error for the semi major axis via Gibbs method for an eccentricity of zero.}
		\label{fig:circulargibbs}
	\end{figure}
		It should be noted that the figure has been cropped to have a Z axis limit of $10^5$ kilometers. This is done so that the color mapping shows the change in the method more clearly. From this figure it can be seen that Gibbs method is incredibly inaccurate for position vectors taken within 10 degrees of each other. This is likely due to the noise attached with the position measurements. Gibbs method is already not intended to work with close position vectors. Thus the error that causes a small problem when the potions vectors are 30 degrees apart causes a much larger problem when the vectors are only one 5 degrees apart. The RMS error with Herrick-Gibbs methods is seen in figure \ref{fig:circularherrickgibbs}.
		
	%	\section{NOTE TO PHILIP2}
	\begin{figure}[H]
		\centering
		\includegraphics[width=0.7\linewidth]{circularHerrickGibbs}
		\caption{RMS error for the semi major axis via Herrick-Gibbs method for an eccentricity of zero.}
		\label{fig:circularherrickgibbs}
	\end{figure}
	

	Unlike Gibbs, Herrick Gibbs excels in the very close range, as it is designed to do. It dips to it's lowest error at a separation of 10 degrees, twice the range Vallado  \cite{vallado2007fundamentals} said it could be used for. Figure \ref{fig:bestmethodscirc} compares both methods.

%\section{NOTE TO PHILIP}
%\textcolor{red}{GIBBS is so bad here because the vectors might be close enough that they are within the noise level!!!!!!!!!!!!!!!!!!!!!!!!!!!!!!!}
\begin{figure}[H]
	\centering
	\includegraphics[width=0.7\linewidth]{bestMethodsCirc}
	\caption{The best possible RMS error for the semi major axis. The red transition line indicates where Gibbs method becomes better than the Herrick Gibbs Method}
	\label{fig:bestmethodscirc}
\end{figure}
	Figure \ref{fig:bestmethodscirc} shows the lowest error of the two methods. It starts by showing the error of Herrick-Gibbs, then at the red line it transitions over to Gibbs method. From this figure Gibbs method is seen to overtake Herrick-Gibbs at a separation of 15 degrees for low noise levels,while at high noise levels it takes up to 35 degrees of separation for Gibbs method to overtake Herrick-Gibbs. From here the error associated with both methods over a range of eccentricity will be analyzed.\par 
	
	\iffalse
	
	
	For this orbit both methods were run given a range of measurement separation and noise.  over a range of degrees . They were subject to noise
	
	The purpose of this report is to perform an analysis of the Gibbs and Herrick-Gibbs methods of orbital determination when subject to noise. The methods are explained and their derivations given. They are then computed for a sample orbit at various eccentricities. The Root Mean Square Error of both methods is obtained and the accuracy of the methods compared.   and sample sizes. 
	
	
	Going to do a statistical analysis with various eccentricities
	This should be before gibbs
	Then Compare Herrick-Gibbs
	The semi Major will be used
	Or I could use the v2
	I think I should just cut down the size untill I don't have to deal with the asymtote anymore
	Look at what the fund book says for when they are good or no
	
	\section{Results}
	Explain I have a staring orbit
	I go through it for diffrent mean anamolys
	With a a noise
	Then do this for diffrent eccentricities
	Plotting the best of both
	Use the v2 norm, or the a norm?
	Then Look at the change in which method is the most accurate
	
	Purpose of this si not so much the mean anamoly change, but the amount of noise and samples?
	\fi
	
	
	\subsection{Analysis of Gibbs Method at various eccentricities}
	The following plots come from Gibb's method at an eccentricity of $e=0.05,0.1,0.15$
	\begin{figure}[H]
		\centering
		\includegraphics[width=0.7\linewidth]{gibbs_e_05}
		\caption{Gibbs method for an eccentricity of 0.05 vs noise and distance between measurements.}
		\label{fig:gibbse05}
	\end{figure}

\begin{figure}[H]
	\centering
	\includegraphics[width=0.7\linewidth]{gibbs_e_05_side}
	\caption{Section view of figure \ref{fig:gibbse05} for specified noise levels.}
	\label{fig:Sidegibbse05}
\end{figure}

	\begin{figure}[H]
	\centering
	\includegraphics[width=0.7\linewidth]{gibbs_e_1}
	\caption{Gibbs method for an eccentricity of 0.1 vs noise and distance between measurements.}
	\label{fig:gibbse1}
\end{figure}

\begin{figure}[H]
	\centering
	\includegraphics[width=0.7\linewidth]{gibbs_e_1_side}
	\caption{Section view of figure \ref{fig:gibbse1} for specified noise levels.}
	\label{fig:Sidegibbse1}
\end{figure}

	\begin{figure}[H]
	\centering
	\includegraphics[width=0.7\linewidth]{gibbs_e_15}
	\caption{Gibbs method for an eccentricity of 0.15 vs noise and distance between measurements.}
	\label{fig:gibbse15}
\end{figure}

\begin{figure}[H]
	\centering
	\includegraphics[width=0.7\linewidth]{gibbs_e_15_side}
	\caption{Section view of figure \ref{fig:gibbse15} for specified noise levels.}
	\label{fig:Sidegibbse15}
\end{figure}

	From these plots it can be seen that Gibbs method is sensitive to both noise and position vector separation. The results are what would be expected, as the distance between position vectors increases the resulting error decreases. Gibbs is also sensitive noise (as expected), where the increase in noise increases the error. \par
	
	Figures \ref{fig:Sidegibbse05}, \ref{fig:Sidegibbse1}, and \ref{fig:Sidegibbse15} show the RMS vs the separation at specified noise levels. From these figures it is seen that changing the eccentricity has very little effect on Gibbs method. 

	%\newpage
	\subsection{Analysis of Herrick-Gibbs Method at various eccentricities}
		The following plots come from Herrick-Gibb's method at an eccentricity of $e=0.05,0.1,0.15$
		\begin{figure}[H]
		\centering
		\includegraphics[width=0.7\linewidth]{herrickgibbs_e_05}
		\caption{Herrick-Gibbs method for an eccentricity of 0.05 vs noise and distance between measurements.}
		\label{fig:herrickgibbse05}
	\end{figure}

\begin{figure}[H]
	\centering
	\includegraphics[width=0.7\linewidth]{heck_gibbs_e_05_side}
	\caption{Section view of figure \ref{fig:herrickgibbse05} for specified noise levels.}
	\label{fig:heckgibbse05side}
\end{figure}

	
	\begin{figure}[H]
		\centering
		\includegraphics[width=0.7\linewidth]{herrickgibbs_e_1}
		\caption{Herrick-Gibbs method for an eccentricity of 0.1 vs noise and distance between measurements.}
		\label{fig:herrickgibbse1}
	\end{figure}
	\begin{figure}[H]
		\centering
		\includegraphics[width=0.7\linewidth]{heck_gibbs_e_1_side}
		\caption{Section view of figure \ref{fig:herrickgibbse1} for specified noise levels.}
		\label{fig:heckgibbse1side}
	\end{figure}
	\begin{figure}[H]
		\centering
		\includegraphics[width=0.7\linewidth]{herrickgibbs_e_15}
		\caption{Herrick-Gibbs method for an eccentricity of 0.15 vs noise and distance between measurements.}
		\label{fig:herrickgibbse15}
	\end{figure}

\begin{figure}[H]
	\centering
	\includegraphics[width=0.7\linewidth]{heck_gibbs_e_15_side}
	\caption{Section view of figure \ref{fig:herrickgibbse15} for specified noise levels.}
	\label{fig:heckgibbse15side}
\end{figure}

	From these plots it can be seen that just like Gibbs method, Herrick-Gibbs is sensitive to both noise and position vector separation. The results are what would be expected, as the distance between position vectors increases Herrick-Gibbs gets better up to a point, then resulting error begins to increase. Again, like Gibbs, Herrick-Gibbs is also sensitive to noise as the increase in noise increases the error. \par

Unlike Gibbs, Herrick-Gibbs is sensitive to changes in eccentricity. Figures \ref{fig:heckgibbse05side}, \ref{fig:heckgibbse1side}, and \ref{fig:heckgibbse15side} show the RMS vs the separation at specified noise levels at three different eccentricities. From these figures it is seen that increasing the eccentricity causes the RMS error to increase faster once it has passed it's minimum point.\par 
 

	\subsection{Comparison of methods}
	
	From the previous sections it can be seen  that Gibbs method improves as the distance between observations increases and Herrick-Gibbs method improves up to a point, then decreases in accuracy. It has also been seen that changing the eccentricity has little effect on Gibbs, but a strong effect on Herrick-Gibbs. These methods are now compared against each other in the following figures. 
	\iffalse
	\begin{figure}[H]
		\centering
		\includegraphics[width=0.7\linewidth]{bothMethods_e_05}
		\caption{Both Gibbs and Herrick-Gibbs at an eccentricity of 0.05}
		\label{fig:bothmethodse05}
	\end{figure}
	\begin{figure}[H]
		\centering
		\includegraphics[width=0.7\linewidth]{bestMethods_e_05}
		\caption{The best possible measurement from either method at an eccentricity of 0.05}
		\label{fig:bestmethodse05}
	\end{figure}
\fi
%\iffalse
\begin{figure}[H]
	\centering
	\begin{subfigure}{.5\textwidth}
		\centering
		\includegraphics[width=1.1\textwidth]{bothMethods_e_05}
		\caption{Both Gibbs and Herrick-Gibbs at an eccentricity of 0.05}
		\label{fig:sub1}
	\end{subfigure}%
	\begin{subfigure}{.5\textwidth}
		\centering
		\includegraphics[width=1.1\linewidth]{bestMethods_e_05}
		\caption{Best possible measurement from either method at an eccentricity of 0.05}
		\label{fig:sub2}
	\end{subfigure}
	\caption{Comparison of methods at an eccentricity of 0.05}
	\label{fig:test}
\end{figure}

\begin{figure}[H]
	\centering
	\begin{subfigure}{.5\textwidth}
		\centering
		\includegraphics[width=1.1\textwidth]{bothMethods_e_1}
		\caption{Both Gibbs and Herrick-Gibbs at an eccentricity of 0.1}
		\label{fig:e1}
	\end{subfigure}%
	\begin{subfigure}{.5\textwidth}
		\centering
		\includegraphics[width=1.1\linewidth]{bestMethods_e_1}
		\caption{Best possible measurement from either method at an eccentricity of 0.1}
		\label{fig:e12}
	\end{subfigure}
	\caption{Comparison of methods at an eccentricity of 0.1}
	\label{fig:test2}
\end{figure}

\begin{figure}[H]
	\centering
	\begin{subfigure}{.5\textwidth}
		\centering
		\includegraphics[width=1.1\textwidth]{bothMethods_e_15}
		\caption{Both Gibbs and Herrick-Gibbs at an eccentricity of 0.15}
		\label{fig:e151}
	\end{subfigure}%
	\begin{subfigure}{.5\textwidth}
		\centering
		\includegraphics[width=1.1\linewidth]{bestMethods_e_15}
		\caption{Best possible measurement from either method at an eccentricity of 0.15}
		\label{fig:e152}
	\end{subfigure}
	\caption{Comparison of methods at an eccentricity of 0.15}
	\label{fig:test3}
\end{figure}
%\fi
\iffalse
	\begin{figure}[H]
	\centering
	\includegraphics[width=0.7\linewidth]{bothMethods_e_1}
	\caption{Both Gibbs and Herrick-Gibbs at an eccentricity of 0.1}
	\label{fig:bothmethodse1}
\end{figure}
\begin{figure}
	\centering
	\includegraphics[width=0.7\linewidth]{bestMethods_e_1}
	\caption{The best possible measurement from either method at an eccentricity of 0.1}
	\label{fig:bestmethodse1}
\end{figure}
	\begin{figure}
	\centering
	\includegraphics[width=0.7\linewidth]{bothMethods_e_15}
	\caption{Both Gibbs and Herrick-Gibbs at an eccentricity of 0.15}
	\label{fig:bothmethodse15}
\end{figure}
\begin{figure}
	\centering
	\includegraphics[width=0.7\linewidth]{bestMethods_e_15}
	\caption{The best possible measurement from either method at an eccentricity of 0.15}
	\label{fig:bestmethodse15}
\end{figure}
\fi

It becomes clear that Gibbs method remains robust as eccentricity increases, while Herrick-Gibbs does not handle the changes in eccentricity as well. While a circular orbit had Herrick-Gibbs beat Gibbs method for up to 35 degrees between measurements, the results are now approaching Vallado's measurement of 5 degrees separation as the change point.\par 

Figure \ref{fig:eccecomp} plots the line where Gibbs method becomes better than Herrick-Gibbs. This provides a more clear demonstration of how as the eccentricity increases, the area where Herrick-Gibbs is the best method decreases. 
\begin{figure}[H]
	\centering
	\includegraphics[width=0.7\linewidth]{ecceComp_3}
	\caption{Comparison of transition lines against a plot of the error for a circular Gibb's method for reference.}
	\label{fig:eccecomp}
\end{figure}

\begin{figure}[H]
	\centering
	\includegraphics[width=0.7\linewidth]{topDownEcce_Comp_3}
	\caption{Top down view of figure \ref{fig:eccecomp}.}
	\label{fig:topdowneccecomp3}
\end{figure}





	
	\section{Conclusion}
	Both Gibbs and Herrick-Gibbs method are affected by measurement noise. This should be considered when performing preliminary orbit determination. However it should also be noted that real orbital determination methods take multiple position measurements, obtaining the mean, and lowering the noise level. \par 
	
	Vallado states that \cite{vallado2007fundamentals} the separation point should be five degrees. However this paper has shown that for measurements subject to measurement error, Herrick-Gibbs performs better than Gibbs over a larger range. Additionally the larger the measurement error, the longer that Herrick-Gibbs is the more accurate method. These findings are backed up Kaushick \cite{Kaushick}'s findings.\par 
	However unlike Kashick's findings  this paper shown a relation between eccentricity and the range where Herrick-Gibbs performs better than Gibbs method. From figure \ref{fig:topdowneccecomp3} it is clear that the transition line is highly dependent on the orbit's eccentricity. An eccentric orbit will transition in a region twice the size of  Vallado's suggested region, but a nearly circular orbit will transition much further along.
	
	The relation between the eccentricity and the accuracy of Herrick=Gibbs was not expected. Recalling equation \ref{eqn:HGREF}, the middle term of the equation for the $\mathbf{ v_2 }$ was: 
	\begin{equation}
	\left( \Delta t _ { 32 } - \Delta t _ { 21 } \right) \left[ \frac { 1 } { \Delta t _ { 21 } \Delta t _ { 32 } } + \frac { \mu } { 12 r _ { 2 } ^ { 3 } } \right] \mathbf { r } _ { 2 }
	\end{equation}
	
	When $ \Delta t _ { 21 }$ was very close to $ \Delta t _ { 32 } $ the effect of the $\mathbf{r_2}$ term would cancel out. For a nearly circular orbit $ \Delta t _ { 21 }$ would be very close to $ \Delta t _ { 32 } $. It was expected that removing this term would increase error, however it has had the opposite effect and decreased error. In fact for very nearly circular orbits, a gouging effect occurs, where the accuracy of Herrick-Gibbs method will dip below Gibbs a second time further down the position measurement distance. The author plans to explore these effects further in later papers.
	
	
	
	
	\iffalse
	
	 seen a relation 
	
	
	Both Gibbs and Herrick-Gibbs method are quite robust to measurement noise. It should be noted that real orbital determination methods take multiple position measurements, obtaining the mean, and lowering the noise level. \par 
	
	What is interesting is the effect of Herrick-Gibbs on noisy measurements of nearly circular orbits. At low eccentricities the Herrick-Gibbs performs better than the Gibbs method subject to noise. 
	
	At very low levels of position vector separation Gibbs method has an incredibly high level of error. This is 
	
	IMPOARTANTE
	
	We have verified that the 5 degres works for low nosie.
	Unlike Tthat other guy we have seen that the eccentricitg of an orbit does have an effect on the range where herrick is better than gibbs
	
	
	
	
	\textcolor{red}{ToDo: Finish Conclusions, Upload plots when computer is finished running ,finsh derivation HeckGibb}
		\fi
		%--------------------------------------
		% References
		% -------------------------------------
		
		\bibliographystyle{unsrt}
		\bibliography{ref}

		
		%-----------------------------------------------------------
		% Appendix
		%-----------------------------------------------------------
		\newpage
		\singlespacing
%		\section*{Appendix 1 -derivation of gibbs}
		\section*{Appendix 1 - MATLAB code}
		\addcontentsline{toc}{section}{Appendix}
		
		\lstset{language=Matlab,%
			%basicstyle=\color{red},
			breaklines=true,%
			morekeywords={matlab2tikz},
			keywordstyle=\color{blue},%
			morekeywords=[2]{1}, keywordstyle=[2]{\color{black}},
			identifierstyle=\color{black},%
			stringstyle=\color{mylilas},
			commentstyle=\color{mygreen},%
			showstringspaces=false,%without this there will be a symbol in the places where there is a space
			numbers=left,%
			numberstyle={\tiny \color{black}},% size of the numbers
			numbersep=9pt, % this defines how far the numbers are from the text
			emph=[1]{for,end,break},emphstyle=[1]\color{red}, %some words to emphasise
			%emph=[2]{word1,word2}, emphstyle=[2]{style},    
		}
	%\lstinputlisting{C:/Users/Philip/Documents/GitHub/Thesis/Master_TLE.m}
	\lstinputlisting{C:/Users/Philip/Documents/GitHub/independent_study_fall_2018/Independat-Study-Fall-2018/Gibbs_Heck_master_loop_Latex.m}
	
	%\subsection{Code}
	%\subsection{Master\_TLE.m}
	%\lstinputlisting{C:/Users/Philip/Documents/GitHub/Thesis/Master_TLE.m}
	%\subsection{get\_SATCAT.m}
	%	\lstinputlisting{C:/Users/Philip/Documents/GitHub/Thesis/get_SATCAT.m}
	%	\subsection{get\_TLE\_from\_ID\_Manager.m}
	%\lstinputlisting{C:/Users/Philip/Documents/GitHub/Thesis/get_TLE_from_ID_Manager.m}
	
	%\subsection{get\_TLE\_from\_NorID.m}
	%\lstinputlisting{C:/Users/Philip/Documents/GitHub/Thesis/get_TLE_from_NorID.m}
	%\lstinputlisting{get_SATCAT.m}
	
	%Thanks for Paul McKee who started this template. It seems to have good matlab code viwing
		
	\end{document}
	
